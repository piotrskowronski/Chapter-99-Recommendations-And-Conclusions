 
\section{Optics: importance of high order dispersion} 

 
\section{Beam chamber alignment} 

One of the problems met in CTF3 was reduced acceptance of some sections. 
It was found that the vacuum chamber was installed far from the expected position.
Therefore, a care should be taken to align the chamber to the design orbit.

 
\section{RF bump dispersion} 

Only during optics corrections of the Combiner Ring it was realized that 
the RF bump was fully achromatic only when the beam was passing through centers 
of the quadrupoles in between the deflectors. 
This finding was already taken into account in the CLIC design and already the version
documented in CDR features fully achromatic solutions.

\section{RF deflectors wakefields}

From the first design of the RF deflectors it was known that the beam must pass as close 
as possible to its center, otherwise the induced wakefields perturb the beam. 
Because of space constraints in this region CTF3 provided suboptimal orbit measurements and 
steering capabilities. However, it was possible to setup the orbit thanks to 
measurements of the beam loading on the cavity output coupler (ref to the section).
Another observable was the turn by turn beam energy loss. 
CLIC should feature not only sufficient orbit measurements and steering capabilities,
but also accurate beam loading measurement to independently verify the setup.



\section{Klystron power margin}   
 
In CTF3 the power of the klystrons was likely to degrade in time. 
In case acceleration changes in one of the structures the optics of the linac needs to be readjusted,
which is a time consuming procedure. Therefore, a comfortable margin should be left in the setting
to allow for eventual corrections. Even if one device performs above the specs
it should not be run at the too high level, because in case of failure the replacement 
one is likely to be less performing.

  
