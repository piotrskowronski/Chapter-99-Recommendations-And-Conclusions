\section{Stability} 
\label{sec.Stability}

The CLIC stability requirements are very stringent what is reflected 
in the hardware specifications. However, in CTF3 several unexpected
effects were discovered, for example influence of electromagnetic noise, 
drifts caused by ambient temperature effects due to seasonal weather changes
or long time to return to a steady state after a device trip. 
Naturally, one would prefer to solve these 
issues by improving the hardware, better insulation or 
adaptations of cooling and ventilation. However, these are usually 
expensive in terms of time and cost. On the other hand, these drifts can be
counteracted by adequate feed-back systems. CLIC hardware, control system and 
instrumentation should forecast implementation of feedback systems on any
sensitive item of the accelerator. It implies not only to the closed loops on
the hardware elements itself but also the beam based feedbacks. 
This is because the measurement devices are often also prone to drifts, 
what is specially true for the RF power and phase measurements.
Their calibration is very sensitive to temperature, electromagnetic noise 
or grounding potential changes. For the last item one needs to keep in mind
that the drive beam will be equipped with 1638 klystrons simultaneously
discharging \textasciitilde 400~A beams into their collectors, which are than directed
into the grounding cables. Its effect, and also coupling to the RF power flowing 
through neighboring waveguides, was observed on the RF measurements at CTF3.
It is unlikely that they can be fully insulated, although it would be very desirable,
therefore beam based measurements should be available with sufficient sensitivity
to provide input for drift analysis and for the feedbacks. 

Fast detection and determination of the drift and jitter sources is also paramount.
Therefore, the control system should allow for logging of all the signals.
If the recording of all the pulse shapes turns out to be too demanding for the control system then
at least a set of parameters that would allow to determine signal deviation from the reference signals 
needs to be saved for each pulse. 
In CTF3 mean and $\chi^2$ values were selected for that purpose and full traces were saved periodically every 10-20 minutes.
What follows, the control system or a dedicated software application should enable possibility 
to archive all the machine and beam signals plus all the settings at any time to save a machine state as a reference.
In CTF3 the application that allowed for easy comparison between different archives and online measurements 
turned out to be the main tool for machine operation and studies.

A dedicated software application should online detect any drifts and 
indicate to the operation a signal that needs to be corrected.
Software for stability analysis of the gathered data should be made ready before the first beam
as it will point to the devices that need to be further stabilized. 
It is because machine that is drifting over time is impossible to tune 
and reaching the required beam performance is impossible.

It is very likely that many feedbacks loops will need to be deployed. 
Therefore, a dedicated system (dedicated front-end computers and software framework) 
should be prepared such that they can be easily implemented
and integrated within the control environment.

The following measurements need to be made available:
\begin{enumerate}

\item Drive beam injector instrumentation and RF stability: any beam parameter change at the exit of the injector
      is strongly amplified by fully loaded acceleration, therefore special care needs to be taken in preparation of
      the beam instrumentation in the injector. Beam current, position and phase plus bunch length should 
      be monitored after each accelerating structure. Also beam loadings need to be measured with sufficient 
      accuracy. Any cost saving in this area is likely to deteriorate significantly the machine performance. 
      If possible, it is advised to use solutions providing redundant stability of the RF power for the injector.

\item Beam loading: change of input power amplitude or phase modifies the beam loading,
      therefore the amplitude at the output couplers needs to be measured with high enough accuracy
      to detect 0.05$^\circ$ phase or 0.2\% amplitude change in the beam loading.

\item Temperature measurements: as mentioned before, the temperature changes are likely to cause beam drifts.
      Therefore, it is important to be able to correlate beam signals with the temperature and eventually
      to implement adequate feedbacks. 

\item Dispersion measurement: the CTF3 experience show that in the strong isochronous optics spurious dispersion
      is very easily produced by relatively small orbit or magnet changes.      
      Ability to monitor, and eventually correct, dispersion in an online manner would be definitely 
      a huge advantage for the efficient machine operation. 

\item Energy measurement: in CTF3 the most important beam jitter and drift was of the beam energy and 
      it is very likely it will the case for CLIC. (explain why in the dedicated section). 
      Additionally, the strong isochronous optics coupled with large energy spread 
      does not leave any margin for energy errors.

\item Emittance and Twiss parameters measurements: maintaining the emittance growth within the limit 
      must be respected for lossless drive beam deceleration and correct functioning of CLIC.
      Being able to observe the emittance drifts in an online manner in the most critical locations
      will allow to understand their origin and eventually design a feedback mechanism. 
      
\end{enumerate}

	
Being obvious and very well covered by the CLIC studies, the orbit measurements are not the part of the list above.
However, the orbit is also very likely to drift somewhat. 
Because running DFS or DTS is rather time consuming, implementing a section for orbit feedback 
in the most critical locations is suggested. This would require a section with 
two BPMs separated by a distance big enough to measure orbit angle change with sufficient precision.
It is suggested to place it before
\begin{itemize}
\item stretching and compressing chicanes
\item the delay loop injection
\item the combiner ring injection
\item emittance measurement locations
\item turn around loops
\end{itemize}


\subsection{Other possible items to mention}

The beam stability after the first bend that the beam passed (TL1 bend or DL septa) was getting worse,
even though the transmission and optic control was very good.
One of the suspects ware tails that were visible in some measurements (energy spread, MTV profiles).
On the other hand if the tails were static (the same from shot to shot) the would be lost at the same place 
and would not affect the stability. \textbf{Can we conclude anything about need for colimation at different stages of the DB Complex?}
