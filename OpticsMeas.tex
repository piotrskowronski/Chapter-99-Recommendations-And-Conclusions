\section{Optics measurements}

\subsection{Emittance and Twiss parameters}

Emittance and Twiss parameters measurements are naturally the key in 
a single pass machine setup. In CTF3 they were done using quad-scan technique. 
To obtain a precise measurement several pulses were recorded at each of the quad settings.
Additionally, change of quadrupole setting is rather slow, especially if one would like to
remove the hysteresis effect. In CTF3 measuring emittance for single plane took about 3 minutes.
Measuring emittance behind the combiner ring for the beam making different number of turns,
to observe the beta beating and emittance growth, was an over an hour task.
Therefore, choosing the fastest technique is suggested.

Additionally, the emittance measurements are also affected by dispersion.
Adding a section where dispersion, and also orbit, can be corrected upfront 
of the location of the emittance measurement is recommended.

In CTF3 quad-scans were problematic if the measured beam had Twiss \textalpha~parameter 
outside \textpm 3 in any of the planes. 
In such case finding a range was very difficult because 
covering sufficient phase advance range needed very strong quad settings, 
which usually lead to too large beam size in the other plane and subsequent beam losses.
In case quad-scans are also foreseen at CLIC and small \textalpha~parameter 
can not be guaranteed at the measurement location then 
additional quadrupoles should be added to guarantee sufficient measurement range.
Additionally, we recommend installation of beam intensity monitor just in front
of profile measurement screen to guarantee that no beam is lost during scan. 

\subsection{Dispersion}

Another key optics parameter that needs to be efficiently measured is dispersion.
Because of the relatively large energy spread dispersion errors are the dominant 
contribution for the emittance growth. Its error  modify $R_{56}$ of the transfer lines and lead to
bunch length elongation. Therefore, as precise as possible measurement of dispersion is required.

\subsection{Transfer Matrix and beta-beating}

One of the first items that needs to be checked during commissioning is 
agreement between the model and the actual machine optics.
The basic tool is the response matrix measurement. 
In CTF3 its more advanced and accurate version was used called Phase Space Painting (ref to PhSpP section).
The orbits are changed using to consecutive orbit correctors. In order to avoid uncertainties associated 
with transfer matrix between the correctors it is preferable they are separated by a drift space.
These can be the same correctors which are used for orbit feedback, see~\ref{sec.Stability}. 
Also, the BPMs used for the feedback allow to check that the orbits are painted as required.
In other words, that there is no error in the corrector's excitation constants.
The adequate software tools allowing for efficient measurements and analysis need to be prepared for the commissioning.

Measurement and correction of the optics in the rings poses additional challenges
because in case of non-zero chromaticity and large energy spread the excited orbits are quickly damped to
the closed orbit. Additionally, the beam energy is decreasing due to the synchrotron radiation.
For the same reasons measurement of the tunes is difficult.
Providing a possibility to store the beam in the rings during the commissioning by installing a small
accelerating cavity would tremendously facilitate the process because all the very accurate techniques
routinely used in the circular accelerators could employed. They offer sub percent accuracy, which is 
otherwise would be very difficult to reach. 
