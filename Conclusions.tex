
\chapter{Conclusions}
\label{sec.Conclusions}

All identified feasibility issues of the CLIC two-beam scheme, from drive beam
generation to its deceleration to produce RF power, and to the use of such power for
high gradient acceleration of a second beam have been successfully addressed in CTF3
and were documented in this document. 
\begin{enumerate}
\item Drive Beam Generation                      
  \begin{enumerate}
    \item Full beam loading acceleration	
    \item High current stable acceleration	
    \item Bunch length control, isochronous lines	
    \item Phase coding		
    \item Combination with RF deflectors	
    \item Drive beam stability (phase, charge,...)
  \end{enumerate}
\item RF Power Production
  \begin{enumerate}
   \item RF power level and pulse length (break down limit)	 
   \item Extraction efficiency, High Order Modes,	
   \item Drive Beam deceleration (efficiency, transport, stability)  
   \item On-off mechanism (break-down protection) 	 
   \item RF pulse shape (beam loading compensation)	   
  \end{enumerate}
\item Two-Beam acceleration
  \begin{enumerate}
   \item Gradient, pulse length (break-down limit) 
   \item Consistency with expectations	  
   \item Break-down kicks		  
   \item Test with fledged module	
   \item Wake-field monitors		  
  \end{enumerate}
\end{enumerate}

Building and constructing CTF3 gave indications on cost and performance issues.
(...)


Operating the two beam acceleration in a test facility gave possibility to 
verify the machine setup and optimization procedures. 
Their performance and their shortcomings were listed in this document.
In many cases improved or alternative, and better performing, solutions were worked out. 
The gained experience allowed to precisely define needs for 
\begin{itemize}
\item beam instrumentation and its locations
\item optics design
\item control system functionalities
\item dedicated software tools
\item optics measurement and correction tools
\item most sensitive areas where extra flexibility is required and 
         cost saving solutions are likely to hamper the machine performance
\end{itemize}
All these items are detailed in Chapter~\ref{chap.Recomendations}.


Beam experiments aimed at mitigating technological risks. 
(...)

CTF3 provided a test bed for the CLIC technology
\begin{enumerate}
 \item Two Beam Module
 \item Phase Feed Forward
 \item Beam Instrumentation
\end{enumerate}
In basically all these cases the tests pointed out difficulties that were all eventually
overcome by either improved designs or dedicated techniques.
The details can be found in the respective chapters. 
Nevertheless, it shows importance of the test facility that offered space and time 
for such studies and technology improvements.
Many of these results brought progress to the accelerator technologies in general.
If CLIC is ever built the knowledge gained in CTF3 definitely translates into its better performance.


%\begin{figure}[h]
% \begin{center}
%  \includegraphics[width=0.5\columnwidth]{figs/99/examplefig.eps}
% \end{center}
%\caption{ The relevant caption.
%\label{cap.intro.MyLabel}} 
%\end{figure}




